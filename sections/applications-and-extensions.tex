% cSpell:ignore DBLP FPGAs
% cSpell:ignoreRegExp /textcite{.*}/
% cSpell:ignoreRegExp /cite{.*}/

\section{Applications and extensions}

	\begin{frame}{Oblivious Binary Search Tree}
		
		PathORAM can be used to perform search on an oblivious binary search tree, using $\BigO{\log^2 N}$ bandwidth.~\cite{Gentry:2013}

		\note{
			\textcite{Gentry:2013} suggested that ORAMs can be used to perform search on an oblivious binary search tree.

			Underlying data structure for PathORAM is an oblivious binary tree.
			One \textsc{Access} for the ORAM is equivalent to binary tree search.
			This way, without re-randomization and write back subroutine, PathORAM \textsc{Access} is the same as binary search.
			Thus, the bandwidth is $\BigO{\log^2 N}$.
		}
	\end{frame}

	\begin{frame}{Stateless ORAM}
		
		In order to avoid complicated (and possibly expensive) oblivious state synchronization between the clients, \citeauthor{DBLP:journals/corr/abs-1105-4125} introduce the concept of stateless ORAM~\cite{DBLP:journals/corr/abs-1105-4125} where the client state is small enough so that any client accessing the ORAM can download it before each data access and upload it afterwards.

		\note{
			If we are using recursive PathORAM, we can upload and download client state --- which is $\BigO{\log N} \cdot \omega (1)$ --- before each access.
			This way we can build a \emph{stateless} ORAM --- potentially, multi-user ORAM\@.
		}
	\end{frame}

	\begin{frame}{Secure Processors}
		
		PathORAM is particularly amenable to hardware design because of its simplicity and low on-chip storage requirements.

		\textcite{Maas:EECS-2014-89} built a hardware implementation of a Path ORAM based secure processor using FPGAs and the Convey platform. % chktex 8

		\textcite{Fletcher:2012:SPA:2382536.2382540, fletcher2013ascend} and \textcite{ren2013design} built a simulator for a secure processor based on PathORAM\@.

		\note{
			Due to its simplicity, PathORAM is particularly good for silicon implementations.
			For example, \textcite{Maas:EECS-2014-89} has build such implementation using FPGAs. % chktex 8
			\textcite{Fletcher:2012:SPA:2382536.2382540} and \textcite{ren2013design} built a simulator for a processor based on PathORAM\@.
		}
	\end{frame}

	\begin{frame}{Integrity}
		
		The protocol can be easily extended to provide integrity (with freshness) for every access to the untrusted server storage.

		We can achieve integrity by simply treating the Path ORAM tree as a \emph{Merkle tree} where data is stored in all nodes of the tree.

		\[
			H (b_1 || b_2 || \ldots || b_Z || h_1 || h_2 )
		\]

		\note{
			It is possible to treat PathORAM internal tree structure as a Merkle tree.
			Each node is tagged with the hash of the following form, which is a concatenation of hashes of all blocks in the bucket, and the children of the node.
		}
	\end{frame}

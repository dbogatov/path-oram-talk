% cSpell:ignore bibfile Goldreich Dautrich

\section{Oblivious Memory}

	\begin{frame}{Problem statement}
		
		Untrusted server.
		Secure database --- each record is encrypted.
		What are we missing?

		\pause%

		\begin{alertblock}{Security vulnerability}
			Adversary still sees the \textbf{access pattern} --- an ordered sequence of read/write operations on data records.
		\end{alertblock}

		\begin{exampleblock}{Attack example}
			Compromising Privacy in Precise Query Protocols~\cite{Dautrich:2013:CPP:2452376.2452397}.
			Observe range queries to recover order.
			Requires $10^4$ queries to compromise privacy for database of over $10^6$ records.
		\end{exampleblock}

		\note{
			The idea is to build a secure cloud, and more specifically --- secure database.
			The server is untrusted --- we assume an adversary can read every byte on the disk and track all CPU operations, but cannot interfere.
			This threat model is called \emph{honest-but-curious} adversary.
			The first step is encrypting the database, so that only client can decrypt.
			But that is just the first step --- what are we missing?

			An adversary can see the access pattern.
			It can see which records are accessed more often.
			Which records are accessed only after some other records were touched.
			How ofter read vs write operations occur.

			\emph{What are the examples of attacks when access pattern is leaked?}
		}
	\end{frame}

	\begin{frame}{Oblivious RAM}

		\begin{block}{Definition}
			A machine is \emph{oblivious} if the sequence in which it accesses memory locations is equivalent for any two inputs with the same running time~\cite{Goldreich:1996:SPS:233551.233553}.
		\end{block}

		\note{
			A solution is to design an oblivious memory access system.
			This definition of oblivious machine is cited from the original paper on ORAMs by Goldreich and others from May 1996 --- around my birthday.
			Among the other things the paper states a number of theorems on computational bounds of generic ORAMs.

			Although the paper analyzes generic ORAMs, 20 years ago people were more concerned about CPU working with RAM access patterns.
			The cloud did not really exist at that time.

			So the one purpose of ORAM is to hide the access pattern.
			We will come to more formal security definition in a couple of slides.
		}
	\end{frame}

	\begin{frame}{Theoretical bounds}
		
		\begin{block}{Theorems}
			Let $\RAM(m)$ denote a \RAM\ with $m$ memory locations and access to a random oracle.
			Then $t$ steps of an arbitrary $\RAM(m)$ can be simulated by
			\begin{itemize}
				\item
					at most $\BigO{t \cdot (\log_2 t)^3}$ steps of an oblivious $\RAM(m \cdot (\log_2 m)^2 )$
				\item
					at least $\max \{ m, (t-1) \cdot \log_2 m \}$ steps of an oblivious $\RAM(m)$
			\end{itemize}

			\cite{Goldreich:1996:SPS:233551.233553} % chktex 2

		\end{block}
		
		\note{
			This are some theorems stated and proved in the paper.
			I am not going to do proofs here.
			The idea is that these are theoretical bounds for generic ORAMs.
			Designing our owm ORAM the aim is to come as close as possible to lower bounds.
		}
	\end{frame}

\section{Evaluation}

	\begin{frame}{Stash Occupancy Distribution}
		
		\begin{block}{Experiment setup}
			
			\begin{itemize}
				\item 
					PathORAM uses a binary tree with height $L = \ceil*{\log_2 N} - 1$.
					% TODO: why?
				\item
					\emph{Stash occupancy} is the number of overflowing blocks.
					Represents client's \emph{persistent} local storage.
				\item
					First load $N$ blocks into ORAM and then access each block in a round-robin pattern.
					Worst-case access pattern in terms of stash occupancy.
				\item 
					Single run for about 250 billion accesses after doing 1 billion accesses for warming-up.
			\end{itemize}
			
		\end{block}
			
		\note{
			In the experiments, authors used a binary tree of height $\ceil*{\log_2 N} - 1$.

			Let us define \emph{stash occupancy} is the number of overflowing blocks.
			Thus, this would be a client's persistent local storage in addition to $Z \log_2 N$ transient storage for storing single path.

			We access PathORAM in a round-robin fashion --- it is proven to be the worst-case scenario for PathORAM\@.
			Round-robin fashion is like $\{ 1, 2, \ldots, N, 1, 2, \ldots, N, \ldots, 1, 2, \ldots \}$.

			Finally, they perform around billion accesses for warming-up, then 250 billion accesses for actual experiment.
		}
		
	\end{frame}

	\begin{frame}{Stash Occupancy Distribution}
		
		\begin{block}{Results}
			
			\begin{itemize}
				\item 
					Required stash size grows linearly with the security parameter.
					Failure probability decreases exponentially with the stash size.
					See appendix.
				\item 
					Extrapolate those results for realistic values of $\lambda$.
					See appendix.
				\item 
					It is by definition infeasible to simulate for practically adopted security parameters (e.g., $\lambda = 128$)
				\item
					Required stash size for a low failure probability does not depend on $N$.
					This shows PathORAM has good scalability.
					See appendix.
			\end{itemize}
			
		\end{block}
			
		\note{
			Some of the key observations made after simulation.

			Required stash size grows linearly with the security parameter, which is consistent with the theorem that failure probability decreases exponentially with the stash size.

			This linearity was used to predict required stash size for realistic security parameters.
			Note that it is by definition infeasible to simulate for practically adopted security parameters, since if we can simulate a failure in any reasonable amount of time with such values, they would not be considered secure.

			Finally, it is discovered that required stash size does not depend on $N$, which proves that PathORAM is highly scalable.
		}
		
	\end{frame}

	\begin{frame}{Bucket Load}
		
		\begin{block}{Results}
			
			\begin{itemize}
				\item 
					For $Z \ge 5$, the usage of a subtree is close to the number of buckets in it.
					Also holds for $Z = 4$.
					% TODO why?
				\item 
					For the levels close to the root, the bucket load is indeed 1 block.
				\item 
					Leaves have slightly heavier loads as blocks accumulate at the leaves of the tree.
				\item
					$Z = 3$, however, exhibits a different distribution of bucket load and produces much larger stash sizes in practice.
			\end{itemize}
			
		\end{block}
			
		\note{
			Another set of observations about bucket load.
			
			For $Z \in \{ 4, 5 \}$, the usage of a subtree is close to the number of buckets in it.

			For the levels close to the root, the bucket load is 1 block.

			Leaves have slightly heavier loads as blocks accumulate at the leaves of the tree.

			$Z = 3$ is different.
			It produces much larger stash sizes in practice, so should not be used in production.
		}
		
	\end{frame}

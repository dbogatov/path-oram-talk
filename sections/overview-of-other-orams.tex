\section{Overview of other ORAMs}

	\begin{frame}{ORAMs Experimental Evaluation}
		
		\begin{tabular}{ l c c c c }

			\toprule%

			$ORAM$				& Computation	& Communication	& Server		& Client															\\

			\midrule%

			Basic-SR			& $N \log N$	& $N \log N$	& $N$			& $1$																\\
			IBS-SR				& $N$			& $\sqrt{N}$	& $N$			& $\sqrt{N}$														\\
			Basic-HR			& $N \log^2 N$	& $N \log^2 N$	& $N \log N$	& $1^b$																\\
			BB-ORAM				& $\log^2 N$	& $N \log^2 N$	& $N \log N$	& $1$																\\
			TP-ORAM				& $\sqrt{N}$	& $1$			& $N$			& $\sqrt{N} + \frac{N}{B}$											\\
			\textbf{Path-ORAM}	& $\bm{\log N}$	& $\bm{1}$		& $\bm{N}$		& \footnote{$\BigO{\log N} \cdot \omega(1) + \BigO{\frac{N}{B}}$}

		\end{tabular}

		Table 2 from~\cite{Chang:2016:ORD:2994509.2994528}.
		Worst-case scenarios shown.

		\note{
			Chang and others published a great paper a year ago doing accurate comparison of known ORAM systems.
			They analyzed space and time complexity of the systems.
			The result is on the table.

			Computational overhead is a composite of communication, encryption/decryption and client running overheads.
			Communication overhead measures how much information is transmitted during a round of read and write.
			Server and client respectively show how much space is used by an ORAM on the server and on the client.

			I am not going to elaborate on all ORAMs, but it is clear that one of them wins in every category.
			This is why we have chosen it for our secure cloud.
		}
	\end{frame}

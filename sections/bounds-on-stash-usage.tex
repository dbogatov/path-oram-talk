\section{Bounds on stash usage}

	\begin{frame}{Probability of failure is negligible}
		
		\begin{block}{Main theorem}

			Let $\bm{a}$ be any sequence of block addresses with a working set of size at most $N$. 
			For a bucket size $Z = 5$, tree height $L = \ceil{\log N}$ and stash size $R$, the probability of a PathORAM failure after a sequence of load/store operations corresponding to $\bm{a}$, is at most
			
			\[
				\Pr \left[ \text{st} \left( \text{ORAM}_L^5 [ \bm{s} ] \right) > R \; | \; a( \bm{s} ) = \bm{a} \right] \le 14 \cdot (0.6002)^R
			\]

			where the probability is over the randomness that determines $\bm{x}$ and $\bm{y}$ in $\bm{s = (a, x, y)}$.

		\end{block}

		\note{
			The whole proof of negligible failure probability is about proving this theorem.
			It might look complex, but simply put, what it says is that for any ORAM with 5 blocks per bucket, stash usage exceeds some stash size $R$ with probability at most exponentially small with respect to $R$.

			Or, even simpler, the probability of exceeding stash capacity decreases exponentially with the stash size, given that the bucket size $Z$ is large enough.

			The authors prove this theorem in three steps.
		}
	\end{frame}

	\begin{frame}{Probability of failure is negligible}
		
		\begin{block}{Proof outline}
			
			\begin{enumerate}
				\item 
					Introduce a second ORAM, called $\infty$-ORAM
				\item 
					Characterize the distributions of real blocks over buckets in a $\infty$-ORAM for which post-processing leads to a stash usage $> R$
				\item 
					Analyze the usage of subtrees $T$ (introduced in the proof)
			\end{enumerate}

		\end{block}

		\note{
			First, they introduce a second ORAM, called $\infty$-ORAM, together with an algorithm that post-processes the stash and buckets of $\infty$-ORAM in such a way that the blocks over buckets distribution of $\infty$-ORAM and real ORAM are the same.

			Second, they show that the stash usage after post-processing is $> R$ if and only if there exists a subtree $T$ for which its ``usage'' in $\infty$-ORAM is more than its ``capacity''.

			Finally, they show how a mixture of binomial and geometric probability distributions expresses the probability of the number of real blocks that do not get evicted from a subtree after a sequence of load/store operations.
		}
	\end{frame}
